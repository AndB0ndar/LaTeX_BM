\graphicspath{{./fourth/img/}}

\section{Цель практической работы}
\textbf{Цель занятия:}
Ознакомление с функциональными возможностями программного обеспечения
по создание бизнес-моделей (процессов, осуществляемых различными сотрудниками
и отделами организаций(предприятий, учереждений)) в методологии IDEF0.\par

\section{Выполнение практической работы}
\subsection{Построение котексной диаграммы}
В качестве процесса предлагается процесс управления
информационным взаимодействием.

\begin{image}
	\includegrph{Screenshot from 2023-09-19 16-25-40}
	\caption{Декомпозиция "<Управлять информационным взаимодействием">}
	\label{fig:IDEF0:a0}
\end{image}

\begin{image}
	\includegrph{Screenshot from 2023-09-21 11-17-17}
	\caption{Декомпозиция "<Управлять информационным взаимодействием">}
	\label{fig:IDEF0:a0:d}
\end{image}

\begin{image}
	\includegrph{Screenshot from 2023-09-19 16-25-51}
	\caption{Декомпозиция "<Осуществить реализацию арестованного имущества">}
	\label{fig:IDEF0:a4}
\end{image}

\subsection{Формирование таблиц}
Сформируем таблицу, где необходимо указать все Входы, Выходы, Механизмы и
Управление.
\leftline{Таблица 1. Элемент нотации IDEF0}
{\small
\begin{longtable}{ |p{0.23\textwidth}
	|p{0.17\textwidth}
	|p{0.17\textwidth}
	|p{0.15\textwidth}
	|p{0.2\textwidth}
	| } 
	\hline
	\multirow{2}{0.23\textwidth}{\textbf{Наименование диаграммы/код}}
		& \multicolumn{4}{|c|}{\textbf{Элемент нотации IDEF0}} \\ \cline{2-5}
		& \textbf{Вход} & \textbf{Выход}
		& \textbf{Механизм} & \textbf{Управление} \\ \hline
	\endhead
	Внести постановление о возбуждении исполнительного производства/A1
		& Исполнитель-ный лист/I1
		& Постановление о возбуждении исполнительного производства/O1
		& Судебный пристав/M1
		& Федеральный закон от 27.07.2003 №79-ФЗ
			"<О государственной гражданской службе РФ">/C1,
			Федеральный закон от 21.07.1991 №118-ФЗ
				"<О судебных приставах">/C2,
			Федеральный закон от 02.10.2007 №229-ФЗ
				"<Об исполнительном производстве">/C3 \\ \hline 
	Проверить имущественное положение должника
		(розыск и арест счетов должника)/A2
		& Инофрмация о считах должника/I2
		& Постановленипе о наложении ареста/O2
		& Судебный пристав/M1
		& Постановление о возбуждении исполнительного
			производства/C4 \\ \hline 
	Отразить действие в документации/A3
		& Постановленипе о наложении ареста/I3
		& Постановление о реализации имущества/O3
		& Судебный пристав/M1
		& Федеральный закон от 27.07.2003 №79-ФЗ
			"<О государственной гражданской службе РФ">/C1,
			Федеральный закон от 21.07.1991 №118-ФЗ
				"<О судебных приставах">/C2,
			Федеральный закон от 02.10.2007 №229-ФЗ
				"<Об исполнительном производстве">/C3 \\ \hline 
	Осуществить реализацию арестованного имущества/A4
		& Постановление о реализации имущества/I4
		& Документ на предостваление в банк/O4
		& Судебный пристав/M1
		& Федеральный закон от 21.07.1991 №118-ФЗ
			"<О судебных приставах">/C1,
			Федеральный закон от 02.10.2007 №229-ФЗ
			"<Об исполнительном производстве">/C2 \\ \hline 
	Оценить имущество/A41
		& Постановление о реализации имущества/I5
		& Оценочная стоимость имущества/O5
		& Судебный пристав/M1
		& Федеральный закон от 21.07.1991 №118-ФЗ
			"<О судебных приставах">/C1,
			Федеральный закон от 02.10.2007 №229-ФЗ
			"<Об исполнительном производстве">/C2 \\ \hline 
	Выставить имущество на торги/A42
		& Оценочная стоимость имущества/I6
		& Денежный эквивалент реализованного имущества/O6
		& Судебный пристав/M1
		& Федеральный закон от 21.07.1991 №118-ФЗ
			"<О судебных приставах">/C1,
			Федеральный закон от 02.10.2007 №229-ФЗ
			"<Об исполнительном производстве">/C2 \\ \hline 
	Зачислить на счет подразделения/A43
		& Денежный эквивалент реализованного имущества/I7
		& Определенная сумма для погашения долга взыскателя
			и остаточная сумма реализации/O7
		& Судебный пристав/M1
		& Федеральный закон от 21.07.1991 №118-ФЗ
			"<О судебных приставах">/C1,
			Федеральный закон от 02.10.2007 №229-ФЗ
				"<Об исполнительном производстве">/C2 \\ \hline 
	Закрыть долг взыскателя/A44
		& Определенная сумма для погашения долга взыскателя
			и остаточная сумма реализации/I8
		& Документ на предостваление в банк/o8,
			Сумма для погащения расходов/O8
		& Судебный пристав/M1
		& Федеральный закон от 21.07.1991 №118-ФЗ
			"<О судебных приставах">/C1,
			Федеральный закон от 02.10.2007 №229-ФЗ
				"<Об исполнительном производстве">/C2 \\ \hline 
	Возместить расходы по совершению исполнительных действий/A45
		& Сумма для погащения расходов/I9
		& Документ на предостваление в банк/O9
		& Судебный пристав/M1
		& Федеральный закон от 21.07.1991 №118-ФЗ
			"<О судебных приставах">/C1 \\ \hline 
\end{longtable}
}

Выявим такие типы связей, как "<вход-выход">, "<обратная связь по входу">, 
"<обратная связь по управлению">, "<управление">, "<выход-механизм">
и составим их список в таблице.

\leftline{Таблица 2. Типы связей}
{\small
\begin{longtable}{
		|p{0.3\textwidth}
		|p{0.3\textwidth}
		|p{0.3\textwidth}
		| } 
	\hline
	\textbf{Наименование диаграммы/код}
		& \textbf{Наименование потока}
		& \textbf{Тип связи} \\ \hline
	\endhead
	Внести постановление о возбуждении исполнительного производства/A1
		& Постановление о возбуждении исполнительного производства
		& Управление \\ \hline
	Проверить имущественное положение должника
		(розыск и арест счетов должника)/A2
		& Постановленипе о наложении ареста
		& Вход-выход \\ \hline
	Отразить действие в документации/A3
		& Постановление о реализации имущества
		& Вход-выход \\ \hline
	Оценить имущество/A41
		& Оценочная стоимость имущества
		& Вход-выход \\ \hline
	Выставить имущество на торги/A42
		& Денежный эквивалент реализованного имущества
		& Вход-выход \\ \hline
	Зачислить на счет подразделения/A43
		& Определенная сумма для погашения долга взыскателя
			и остаточная сумма реализации
		& Вход-выход \\ \hline
	Закрыть долг взыскателя/A44
		& Сумма для погащения расходов
		& Вход-выход \\ \hline
\end{longtable}
}

Определим объект преобразования по типу: информационный и материальный,
и составим таблицу.

\leftline{Таблица 3. Типы объектов}
{\small
\begin{longtable}{
		|p{0.3\textwidth}
		|p{0.3\textwidth}
		|p{0.3\textwidth}
		| } 
	\hline
	\textbf{Элемент нотации IDEF0}
		& \textbf{Наименование преобразуемого объекта}
		& \textbf{Тип(информационный, материальный)} \\ \hline
	\endhead
	Вход & Исполнительный лист & информационный \\ \hline
	Внутренний поток
		& Постановление о возбуждении исполнительного производства
		& информационный \\ \hline
	Внутренний поток & Постановленипе о наложении ареста
		& информационный \\ \hline
	Внутренний поток (Выход) & Постановление о реализации имущества
		& информационный \\ \hline
	Внутренний поток (Выход) & Документ на предостваление в банк
		& информационный \\ \hline
	Внутренний поток & Постановление о реализации имущества
		& информационный \\ \hline
	Внутренний поток & Оценочная стоимость имущества
		& информационный \\ \hline
	Внутренний поток & Денежный эквивалент реализованного имущества
		& информационный \\ \hline
	Внутренний поток & Определенная сумма для погашения долга взыскателя
			и остаточная сумма реализации
		& информационный \\ \hline
	Внутренний поток & Сумма для погащения расходов
		& информационный \\ \hline
\end{longtable}
}

\section*{ВЫВОД}
\addcontentsline{toc}{section}{ВЫВОД}
В результате получили построенные и сохраненные в файле текстового формата
дерево узлов процесса, структурно-функциональная диаграмма бизнес-процесса,
таблицы.

\section*{СПИСОК ЛИТЕРАТУРЫ}
\addcontentsline{toc}{section}{СПИСОК ЛИТЕРАТУРЫ}
\begin{thebibliography}{}
    \bibitem{}  Материалы для практических/семинарских занятий: [url] 
		\url{https://online-edu.mirea.ru/mod/resource/view.php?id=496092}
\end{thebibliography}
