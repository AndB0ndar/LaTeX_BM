\graphicspath{{./14/img/}}

\section{Цель практической работы}
\textbf{Цель занятия:}
отработка применения типизации событий и элемента
«Задача», а также маркеров действий при создании моделей процессов в
методологии BPMN.

\section{Выполнение практической работы}

\textbf{Цель занятия:} отработка применения типизации событий и элемента 
«Задача», а также маркеров действий при создании моделей процессов в 
методологии BPMN. 

\textbf{Постановка задачи:} 
на основе выданного преподавателем варианта задания:
\begin{itemize}
	\item сформировать текстовое описание бизнес-процесса, определив роли 
		(исполнителей), инициирующее и завершающее событие, тем самым 
		определив границы бизнес-процесса;
	\item построить бизнес-процесс в нотации BPMN,
	\item подготовить презентацию для публичной защиты бизнес-процесса, 
		защитить полученную модель.
\end{itemize}

\textbf{Результат практического занятия:} построенные и сохраненные в 
файле текстового формата текстовое описание бизнес-процесса, модели 
бизнес-процесса, презентация бизнес-процесса, представленные 
преподавателю в конце практического занятия в виде отчета. Студентам 
также рекомендуется сохранить файл с процессом в формате png для 
дальнейшей работы с ним на другом практическом занятии.

\subsection{Задание}
При построении модели бизнес-процесса согласно выданному варианту 
необходимо обеспечить как минимум два пула с целью получения 
диаграммы взаимодействия. Один из пулов может представлять собой либо 
контрагента, либо другой процесс. Все элементы «Задача» и события студент 
должен типизировать, а также использовать маркеры действий. Применить 
не менее одного раза маркер подпроцесса и построить для него отдельно в 
дальнейшем развёрнутый пул. В процессах кроме их наименования даны 
опорные задачи, которые могут выступить в качестве подпроцесса. При 
недостатке информации студенту рекомендуется обратиться к свободным и 
доступным источникам в Интернете. Варианты процессов (выдает 
преподаватель, каждый студент получает свой вариант): 

\textbf{Процесс}: Оформить ипотечный кредит
\begin{itemize}
	\item обработать заявку клиента
	\item проверить кредитоспособность клиента
	\item выдать ипотечный кредит
\end{itemize}

\subsection{Модель процесса}
Построим бизенс-процесс по заданному в методическом пособии варианту.
Оформление ипотечного кредита осуществляет 
менеджер кредитования и банковский сотрудник.
Контрагентом выступает клиент.
Инициирующее событие --- это получение заявки от клиента.
А завершающее --- выдача ипотечного кредита.

\begin{image}
	\includegrph{2023-11-17\_21-41-40}
	\caption{Процесс "<Оформление ипотечного кредита">}
\end{image}

\begin{image}
	\includegrph{2023-11-15\_19-34-54}
	\caption{Подпроцесс "<Проверка кредитоспособности">}
\end{image}

\clearpage

\section*{ВЫВОД}
\addcontentsline{toc}{section}{ВЫВОД}
В результате практической работы \No 14 мы построили и сохранили в
файле текстового формата текстовое описание бизнес-процесса, модель
бизнес-процесса, презентацию бизнес-процесса.

\clearpage

\section*{СПИСОК ЛИТЕРАТУРЫ}
\addcontentsline{toc}{section}{СПИСОК ЛИТЕРАТУРЫ}
\begin{thebibliography}{}
    \bibitem{}  Материалы для практических/семинарских занятий: [url] 
		\url{https://online-edu.mirea.ru/mod/resource/view.php?id=496092}
\end{thebibliography}
