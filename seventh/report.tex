\graphicspath{{./seventh/img/}}

\section{Построение таблиц с характеристиками}

\leftline{Таблица 1. Характеристики процесса}
{\small
\begin{longtable}{
		|p{0.15\textwidth}
		|p{0.15\textwidth}
		|p{0.15\textwidth}
		|p{0.15\textwidth}
		|p{0.15\textwidth}
		|p{0.15\textwidth}
		| } 
	\hline
	\textbf{Название процесса}
		& \textbf{Цель}
		& \textbf{Владелец}
		& \textbf{Участник}
		& \textbf{Вход}
		& \textbf{Выход} \\ \hline
	\endhead
	Планирование
		& Сформировать план разработки ПО
		& Менеджер проекта
		& Технолог разработки ПО и заказчик
		& Договор на разработку ПО
		& План \\ \hline
	Формирование требований
		& Сформировать требования к ПО
		& Менеджер проекта
		& Технолог разработки ПО и бизнес-аналитик, заказчик ПО и пользователи
		& План
		& Спецификация \\ \hline
	Анализ и проектирование
		& Создать дизайн и архитектуру приложения
		& Менеджер проекта
		& Проектировщие, бизнес-аналитик, разработчик,
			пользователи и заказчик ПО
		& Спецификация
		& Дизайн \\ \hline
	Конструирование
		& Создание ПО
		& Менеджер проекта
		& Разработчик и инженер по качеству и технолок ПО
		& Дизайн
		& Код \\ \hline
	Интеграция и тестирование
		& Протестировать приложение и интегрировать в рабочую область
		& Менеджер проекта
		& Тестировщик, технический писатель,
			разработчик и инженер по качеству
		& Код
		& Продукт \\ \hline
	Поддержка и эксплуатация
		& Эксплуатация приложения до конци жизненного цикла приложения
		& Менеджер проекта
		& Сотрудник сервисного отдела, закачик ПО
			и менеджер по работе с клиентами
		& Продукт
		& \\ \hline
\end{longtable}

\section{Формирование названий выходов каждого
	этапа указанной каскадной модели}
Из процесса "<Исследование концепции"> выходит "<Список требований">.
Из процесса "<Выборка требований"> выходит "<Техническое задание">.
Из процесса "<Проектирование"> выходит "<План разработки">.
Из процесса "<Реализация компонента"> выходит "<Компонент">.
Из процесса "<Интеграция компонента"> выходит "<Документация к компоненту">.

\section{Определение выходов бизнес-процесса "<Разработка ЭИС">}
Из подпроцесса "<Разработка и утверждение технического задания">
выходит "<Техническое задание">.

Из подпроцесса "<Изучение литературы по теме">
выходит "<Список идей возможностей программы">.
Из подпроцесса "<Разработка структуры данный">
выходит "<Структура данных">.
Из подпроцесса "<Выбор метода решения задачи">
выходит "<Метод решения">.
Из подпроцесса "<Разработка алгоритма решения задачи">
выходит "<Алгоритм решения">.
Из подпроцесса "<Выбор среды программирования">
выходит "<Выбранная среда программирования">.
Из подпроцесса "<Согласование и утверждение экскизного проекта">
выходит "<Эскизный проект">.

Из подпроцесса "<Утачнение структуры и формы данных">
выходит "<Структура и форма данных">.
Из подпроцесса "<Разработка модулей программы">
выходит "<Архитектура программы">.
Из подпроцесса "<Согласование и утверждение технического проекта">
выходит "<Технический проект">.

Из подпроцесса "<Программирование объектов"> выходит "<Код">.
Из подпроцесса "<Работа по дополниельной настройке">
выходит "<Исправленный код">.
Из подпроцесса "<Отладка взаимосвязей объектов">
выходит "<Рабочий проект">.

Из подпроцесса "<Работа по исправлению ошибок">
выходит "<Исправленная программа">.
Из подпроцесса "<Повторное тестирование"> выходит "<Результаты тестов">.
Из подпроцесса "<Испытание программы на различных примерах,
анализ результата"> выходит "<Отчет о результатах тестов">
и "<Отлаженная программа">.

Из подпроцесса "<Подготовка программной документации">
выходит "<Документация">.
Из подпроцесса "<Передача документации и программы для внедрения">
выходит "<Документация"> и "<Программа">.

\section{Определение выходов бизнес-процесса "<Разработка ЭИС">}
Из подпроцесса "<Изучение бизнес-процессов предприятия">
выходит "<Список бизнес-процессов">.
Из подпроцесса "<Сжатия бизнес-процессов">
выходит "<Список информации о бизнес-процессах">.
Из подпроцесса "<Сбор требований будущих пользователей программной среды">
выходит "<Список требований программной среды">.
Из подпроцесса "<Написание технического задания на разработку">
выходит "<Техническое задание">.

Из подпроцесса "<Выбор средства разрабоки"> выходит "<Средство разрабоки">.
Из подпроцесса "<Описание клиентской части ЭИС">
выходит "<Описание клиентской части ЭИС">.
Из подпроцесса "<Выбор СУБД"> выходит "<Выбранная СУБД">.
Из подпроцесса "<Описать дерево функций"> выходит "<Дерево функций">.
Из подпроцесса "<Описать сценарий диалога для каждого пользователя">
выходит "<Сценарии диалогов">.
Из подпроцесса "<Описать структурную схему ЭИС">
выходит "<Описание структурной схемы">.
Из подпроцесса "<Описать реализацию БД ЭИС">
выходит "<Описание реализации БД">.
Из подпроцесса "<Создание макетов экранных форм">
выходит "<Макеты экранных форм">.
Из подпроцесса "<Описать технологию работы"> выходит "<Архитектура ЭИС">.

Из подпроцесса "<Создание дизайна для экранных форм">
выходит "<Дизайн для экранных форм">.
Из подпроцесса "<Кодирование модулей"> выходит "<Код модулей">.
Из подпроцесса "<Тестирование модулей"> выходит "<Результаты тестирования">.
Из подпроцесса "<Отладка ЭИС"> выходит "<Олаженый продукт">.
Из подпроцесса "<Документирование"> выходит "<Документация">.

Из подпроцесса "<Внедрение ЭИС в компании-заказчика">
выходит "<Отчет о внедрении">.

\section*{ВЫВОД}
\addcontentsline{toc}{section}{ВЫВОД}
В результате практической работы \No 7 мы научились определять выоды
процессов. А также в целом, описывать подпроцессы в бизнес-процессах.

\section*{СПИСОК ЛИТЕРАТУРЫ}
\addcontentsline{toc}{section}{СПИСОК ЛИТЕРАТУРЫ}
\begin{thebibliography}{}
    \bibitem{}  Материалы для практических/семинарских занятий: [url] 
		\url{https://online-edu.mirea.ru/mod/resource/view.php?id=496092}
\end{thebibliography}
