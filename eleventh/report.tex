\graphicspath{{./eleventh/img/}}

\section{Цель практической работы}
\textbf{Цель занятия:}
Отработка навыков по созданию моделей процессов в методологии BPMN.

\section{Выполнение практической работы}
\subsection{Модель процесса по сказке "<Репка">.}
Построим бизенс-процессы, указанные в методическом пособии, и исправим ошибки,
допущенные при моделировании.

\begin{image}
	\includegrph{eleventh}
	\caption{Измененный процесс "<Обеспечить оплату счета поставщика">}
\end{image}

\begin{image}
	\includegrph{eleventh2}
	\caption{Измененный процесс "<Обработать заказ клиента">}
\end{image}

\section*{ВЫВОД}
\addcontentsline{toc}{section}{ВЫВОД}
В результате практической работы \No 11 мы улучшили навыки
создании моделей процессов в методологии BPMN.

\section*{СПИСОК ЛИТЕРАТУРЫ}
\addcontentsline{toc}{section}{СПИСОК ЛИТЕРАТУРЫ}
\begin{thebibliography}{}
    \bibitem{}  Материалы для практических/семинарских занятий: [url] 
		\url{https://online-edu.mirea.ru/mod/resource/view.php?id=496092}
\end{thebibliography}
