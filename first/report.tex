\graphicspath{{./first/img/}}

\section{Цель практической работы}
\textbf{Цель занятия:} построение функциональной диаграммы процесса и
семантические и логические ошибки в построении функциональной диаграммы.E\par
\textbf{Постановка задачи:} выявить ошибки, допущенные при построении
функциональной диаграммы процесса.\par

\section{Выполнение практической работы}
\subsection{Формирования Технического проекта (ТП)}

На вход Формирование Технического проекта (ТП) приходят:
\begin{itemize}
	\item Утвержденный ЭП
	\item Уточненное ТЗ
	\item Уточняющие данные
\end{itemize}

Механизмами являются:
\begin{itemize}
	\item Уточняющие данные
	\item Программист
	\item Проектировщик
\end{itemize}

Управлением является стандарт.

На выходе будет технический проект.


\begin{image}
	\includegrph{2023-09-02\_09-46-46}
	\caption{Контекстная диаграмма "<Формирования~Технического проекта~(ТП)">}
	\label{fig:IDEF0:a0}
\end{image}

На следующим уровне мы имеем 4 процесса:
\begin{itemize}
	\item Уточнение структуры и формы представления входных и выходных данных
	\item Разработка структуры программы
	\item Определение конфигурации технических средств
	\item Подготовка Пояснительной записки к ТП
\end{itemize}

Подпроцесс «Уточнение структуры и формы представления входных и
выходных данных» осуществляет бизнес-аналитик.
Входами подпроцесса  являются Утвержденное ТЗ,
Утвержденный эскизный проект (ЭП), Уточняющие данные.
Выходом подпроцесса (внутренним потоком) является
Уточненная структура и форма представления данных.\par
Подпроцесс «Разработка структуры программы» осуществляет бизнесаналитик
и программист, используя
Утвержденный ЭП и Уточненную структуру и форму представления данных.
Выходом подпроцесса (внутренним потоком) является Структура программа.\par
Подпроцесс «Определение конфигурации технических средств»
осуществляет программист и проектировщик исходя из Утвержденного ТЗ и
Структуры программы.
Выходом подпроцесса (внутренним потоком) является
Конфигурация технических средств.\par
Затем Бизнес-аналитик на основе Уточненной структуры данных и их
формы, Структуры программы, Конфигурации технических средств реализует
подпроцесс «Подготовка Пояснительной записки к ТП».

\begin{image}
	\includegrph{2023-09-02\_09-53-08}
	\caption{Декомпозиция "<Формирования~Технического~проекта~(ТП)">}
	\label{fig:IDEF0:a1}
\end{image}

\subsection{Изготовление юбки}

На вход Изготовление юбки приходят:
\begin{itemize}
	\item Ткань
	\item Фурнитура
\end{itemize}

Механизмами являются:
\begin{itemize}
	\item Раскройное оборудование
	\item Швея-закройшица
	\item Швейная машинка
	\item Швейновышивальная машинка
\end{itemize}

Управлением является:
\begin{itemize}
	\item Выкройка
	\item Правила изготовления швейного изделия
\end{itemize}

\begin{image}
	\includegrph{2023-09-02\_16-30-22}
	\caption{Контекстная диаграмма "<Изготовление юбки">}
	\label{fig:2:IDEF0:a0}
\end{image}

На выходе будет платье.

Раскрой материала осуществляется с использованием Раскройного оборудования и Выкройки.
В качестве преобразуемого материала выступает Ткань.
Выходом операции являются Детали изделия.

Сшивание деталей изделия осуществляется с использованием Швейной машинки.
Выходом операции является Собранное изделие.

Добавление фурнитуры осуществляется с использованием Швейновышивальной машинки.
Преобразуемыми ресурсами выступают Собранное изделие и Фурнитура (пуговицы, пряжка, аппликация).
Выходом операции является Платье

\begin{image}
	\includegrph{2023-09-02\_16-30-46}
	\caption{Декомпозиция "<Изготовление юбки">}
	\label{fig:2:IDEF0:a0}
\end{image}

\section*{ВЫВОД}
\addcontentsline{toc}{section}{ВЫВОД}
В результате получили построенный без ошибок и сохраненный в файле текстового
формата бизнес-процесс.

\section*{СПИСОК ЛИТЕРАТУРЫ}
\addcontentsline{toc}{section}{СПИСОК ЛИТЕРАТУРЫ}
\begin{thebibliography}{}
    \bibitem{}  Материалы для практических/семинарских занятий: [url] 
		\url{https://online-edu.mirea.ru/mod/resource/view.php?id=496092}
\end{thebibliography}

