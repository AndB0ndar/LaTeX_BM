\graphicspath{{./first/img/}}

\section{Цель практической работы}
\textbf{Цель занятия:} построение функциональной диаграммы процесса и
ознакомление с функциональными возможностями программного
обеспечения.\par
\textbf{Постановка задачи:} построить концептуальную модель и сделать
декомпозицию концептуальной модели на основе текстового описания
процесса.\par
\textbf{Результат практического занятия:} построенные и сохраненные в
файле текстового формата структурно-функциональные диаграммы бизнеспроцессов, представленные преподавателю в конце практического занятия
(форма отчета размещена в СДО).

\section{Выполнение практической работы}
\subsection{Формирования Технического проекта (ТП)}

\begin{image}
	\includegrph{2023-09-02\_09-53-08}
	\caption{Контекстная диаграмма "<Формирования Технического проекта (ТП)">}
	\label{fig:IDEF0:a0}
\end{image}

\begin{image}
	\includegrph{2023-09-02\_09-53-08}
	\caption{Декомпозиция "<Формирования Технического проекта (ТП)">}
	\label{fig:IDEF0:a1}
\end{image}

