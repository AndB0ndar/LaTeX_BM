\graphicspath{{./13/img/}}

\section{Цель практической работы}
\textbf{Цель занятия:}
отработка применения типизации событий и элемента
«Задача», а также маркеров действий при создании моделей процессов в
методологии BPMN.

\section{Выполнение практической работы}
\subsection{Текстовое описание бизнес-процесса}
Процесс начинается с поступления заказа шеф-повару.
Затем, шеф-повар назанчает задачу повару и помощнику повара.
Помошник повара подготавливает ингредиенты. В этот подпроцесс входит
параллельное выполнение замачивания в молоке хлеба, нарезка лука
и прокручивание фарша. Прокручивание фарша выполняется несколько раз,
пока достаточное количество мяса не будет прокручено.
Далее, повар перемешивает ингредиенты в нужных пропорциях и формирует котлеты.
Завершающая работа повара, пожарить котлеты. В конце процесса шеф-повар
сервирует блюдо для подачи.

\subsection{Модель процесса "<Приготовить блюдо под заказ">}
Построим бизенс-процессы, указанные в методическом пособии, и исправим ошибки, допущенные при моделировании.
\begin{image}
	\includegrph{2023-11-11\_10-11-05}
	\caption{Процесс "<Приготовление блюда под заказ">}
\end{image}

\begin{image}
	\includegrph{2023-11-11\_10-11-18}
	\caption{Подпроцесс "<формирование котлет">}
\end{image}

\clearpage

\section*{ВЫВОД}
\addcontentsline{toc}{section}{ВЫВОД}
В результате практической работы \No 13 мы построили и сохранили в
файле текстового формата текстовое описание бизнес-процесса, модель
бизнес-процесса, презентацию бизнес-процесса.

\clearpage

\section*{СПИСОК ЛИТЕРАТУРЫ}
\addcontentsline{toc}{section}{СПИСОК ЛИТЕРАТУРЫ}
\begin{thebibliography}{}
    \bibitem{}  Материалы для практических/семинарских занятий: [url] 
		\url{https://online-edu.mirea.ru/mod/resource/view.php?id=496092}
\end{thebibliography}
