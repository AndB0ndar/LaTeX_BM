\graphicspath{{./third/img/}}

\section{Цель практической работы}
\textbf{Цель занятия:}
построение функциональной диаграммы процесса.\par
\textbf{Постановка задачи:} построить концептуальную модель и сделать
декомпозицию концептуальной модели, провести декомпозицию подпроцессов.\par

\section{Выполнение практической работы}
В качестве процесса предлагается процесс приготовления котлет.
В процессе выполнения практической задания ознакомились
с этапами и последовательности приготовления.
Определены входы и выходы процесса, а также подпроцессов. Выявлены механизмы,
требуемые для приготовления. Связали продпроцессы друг с другом, проверив их
на условие.

\begin{image}
	\includegrph{Screenshot from 2023-09-15 19-58-01}
	\caption{Декомпозиция "<Продажа товаров>}
	\label{fig:IDEF0:a1.3}
\end{image}

\begin{image}
	\includegrph{Screenshot from 2023-09-15 19-57-58}
	\caption{Декомпозиция "<Продажа товаров">}
	\label{fig:IDEF0:a1}
\end{image}

\begin{image}
	\includegrph{Screenshot from 2023-09-15 19-57-55}
	\caption{Контекстная диаграмма "<Продажа товаров">}
	\label{fig:IDEF0:a0}
\end{image}

\section*{ВЫВОД}
\addcontentsline{toc}{section}{ВЫВОД}
Построенные и сохраненные в
файле текстового формата структурно-функциональные диаграммы бизнеспроцессов,
представленные преподавателю в конце практического занятия.

\section*{СПИСОК ЛИТЕРАТУРЫ}
\addcontentsline{toc}{section}{СПИСОК ЛИТЕРАТУРЫ}
\begin{thebibliography}{}
    \bibitem{}  Материалы для практических/семинарских занятий: [url] 
		\url{https://online-edu.mirea.ru/mod/resource/view.php?id=496092}
\end{thebibliography}
