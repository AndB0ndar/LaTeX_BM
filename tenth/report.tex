\graphicspath{{./tenth/img/}}

\section{Цель практической работы}
\textbf{Цель занятия:}
знакомство с функциональными возможностями программного
обеспечения по созданию моделей процессов в методологии BPMN.

\clearpage

\section{Выполнение практической работы}
\subsection{Модель процесса по сказке "<Репка">.}
Построем модель процессов в методологии BPMN по сказке "<Репка">.

\begin{image}
	\includegrph{tenth}
	\caption{Модель процессов по сказке "<Репка">}
\end{image}

\subsection{Модель процесса "<Нанять сотрудника">.}
Построим модель процесса «Нанять сотрудника» в BPMN и проведем его
детализацию.

\begin{image}
	\includegrph{tenth2}
	\caption{Модель процессов "<Найти сотрудника">}
\end{image}

Процесс «Нанять сотрудника» разбивается на следующие подпроцессы:

\begin{enumerate}
	\item Найти кандидатов на вакансию.
	\item Оформить документы.
	\item Обучить нового сотрудника.
\end{enumerate}

Подпроцесс "<Найти кандидатов на вакансию"> осуществляют 4 человека:

\begin{itemize}
	\item Руководитель отдела, который инициирует данный процесс
		путем создания заявки на поиск персонала;
	\item HR-директор;
	\item Генеральный директор организации;
	\item Менеджер по подбору персонала
\end{itemize}

\begin{image}
	\includegrph{tenth2.1}
	\caption{Модель процессов "<Найти кандидатов на вакансию">}
\end{image}

Подпроцесс "<Оформить документы нового сотрудника"> осуществляют 3
человека:

\begin{itemize}
	\item Менеджер по подбору персонала;
	\item Офис-менеджер;
	\item Бухгалтер.
\end{itemize}

\begin{image}
	\includegrph{tenth2.2}
	\caption{Модель процессов "<Оформить документы нового сотрудника">}
\end{image}


Подпроцесс "<Обучить нового сотрудника"> реализуется следующими
участниками:

\begin{itemize}
	\item HR-директор;
	\item Руководитель отдела;
	\item Администратор.
\end{itemize}

Сотруднику выдаются обучающие материалы:

\begin{itemize}
	\item HR-директор выдает материалы о компании;
	\item Руководитель отдела выдает материалы о функциональности отдела;
	\item Администратор обеспечивает доступ к информационным материалам о
		компании.
\end{itemize}

\begin{image}
	\includegrph{tenth2.3}
	\caption{Модель процессов "<Обучить нового сотрудника">}
\end{image}

\clearpage

\section*{ВЫВОД}
\addcontentsline{toc}{section}{ВЫВОД}
В результате практической работы \No 10 мы научились
созданию моделей процессов в методологии BPMN.

\clearpage

\section*{СПИСОК ЛИТЕРАТУРЫ}
\addcontentsline{toc}{section}{СПИСОК ЛИТЕРАТУРЫ}
\begin{thebibliography}{}
    \bibitem{}  Материалы для практических/семинарских занятий: [url] 
		\url{https://online-edu.mirea.ru/mod/resource/view.php?id=496092}
\end{thebibliography}
