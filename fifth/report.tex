\graphicspath{{./fifth/img/}}

\section{Цель практической работы}
\textbf{Цель занятия:}
самостоятельное моделирование бизнес-процесса согласно выданному варианту
в методологии IDEF0.\par
\textbf{Постановка задачи:}
на основе выданного преподавателем варианта построить контекстную диаграмму,
детализацию контекстной диаграммы, детализацию одного из подпроцессов.

\section{Выполнение практической работы}
В качестве процесса предлагается процесс перевозки груза.

\begin{image}
	\includegrph{Screenshot from 2023-09-25 13-50-49}
	\caption{Контекстная диаграмма "<Перевозка груза">}
	\label{fig:IDEF0:a0}
\end{image}

\begin{image}
	\includegrph{Screenshot from 2023-09-28 11-35-18}
	\caption{Декомпозиция контекстной диаграммы "<Перевозка груза">}
	\label{fig:IDEF0:a0:d}
\end{image}

\begin{image}
	\includegrph{Screenshot from 2023-09-28 11-35-46}
	\caption{Декомпозиция подпроцесса "<расчета перевозки груза">}
	\label{fig:IDEF0:a2}
\end{image}

\section*{ВЫВОД}
\addcontentsline{toc}{section}{ВЫВОД}
В результате практической работы получена контекстная диаграмма
по предоставленному варианту.
Бизнес-процесс отражает преобразование материальных в
материальные потоки и сопровождаться, в свою очередь, преобразованием
информационных в информационные потоки.

\section*{СПИСОК ЛИТЕРАТУРЫ}
\addcontentsline{toc}{section}{СПИСОК ЛИТЕРАТУРЫ}
\begin{thebibliography}{}
    \bibitem{}  Материалы для практических/семинарских занятий: [url] 
		\url{https://online-edu.mirea.ru/mod/resource/view.php?id=496092}
\end{thebibliography}
