\graphicspath{{./sixth/img/}}

\section{Цель практической работы}
\textbf{Цель занятия:}
формирование навыка проведения декомпозиции 
процесса в методологии IDEF0.\par
\textbf{Постановка задачи:}
на основе ранее выданного преподавателем 
варианта в практической работе 4:

\begin{enumerate}
	\item Поверить построенную функциональную диаграмму процесса на 
		семантические ошибки. В случае обнаружения ошибок в использовании 
		принципов построения моделей внести исправления в функциональную 
		диаграмму и сформировать текстовый файл, в котором отразить
		все внесенные изменения.
	\item Выбрать любой подпроцесс в декомпозиции бизнес-процесса и 
		построить следующий уровень детализации, руководствуясь тем,
		что входные и выходные потоки, а также механизм управления
		и исполнения уже заданы на более высоком уровне.
		Количество операций в детализируемом подпроцессе не 
		может быть меньше 3 и ограничено 6.
	\item Сформировать табличное описание всех декомпозированных 
		подпроцессов в файле текстового формата.
\end{enumerate}

\section{Выполнение практической работы}
В ходе проверки контекстной диаграммы была выявлена следующая 
ошибка: отсутствует вход подпроцесса
"<Проверить имущественное положение должника
(розыск и арест счетов должника)">.

\begin{image}
	\includegrph{Screenshot from 2023-10-01 20-39-29}
	\caption{Контекстная диаграмма "<Управление информационным
		взаимодействием">}
	\label{fig:IDEF0:a0}
\end{image}

\begin{image}
	\includegrph{Screenshot from 2023-10-01 20-39-33}
	\caption{Декомпозиция контекстной диаграммы "<Управлять информационным
		взаимодействием">}
	\label{fig:IDEF0:a0:d}
\end{image}

\begin{image}
	\includegrph{Screenshot from 2023-10-01 20-39-33}
	\caption{Декомпозиция подпроцесса "<Осуществить реализацию
		арестованного имущества">}
	\label{fig:IDEF0:a4}
\end{image}

\begin{image}
	\includegrph{Screenshot from 2023-10-01 20-39-45}
	\caption{Декомпозиция подпроцесса "<Проверить имущественное
		положение должна (разыск и арест счетов должника)">}
	\label{fig:IDEF0:a2}
\end{image}

\section{Табличное описание всех декомпозированных подпроцессов}

\leftline{Таблица 1. Таблично описание бизнес-процесса
	"<Управлять информационным взаимодействием">
	}
{\small
\begin{longtable}{
		|p{0.15\textwidth}
		|p{0.15\textwidth}
		|p{0.15\textwidth}
		|p{0.15\textwidth}
		|p{0.15\textwidth}
		| } 
	\hline
	\textbf{Название подпроцесса}
		& \textbf{Краткое описание}
		& \textbf{Исполнитель}
		& \textbf{Вход}
		& \textbf{Выход} \\ \hline
	\endhead
	Внести постановление о возбуждении исполнительного производства	
		& Исполнительное производство возбуждается на основании
			исполнительного документа по заявлению взыскателя,
			если иное не установлено настоящим Федеральным законом
		& Судебный пристав 
		& Исполнительный лист 
		& Постановление о возбуждении исполнительного производства \\ \hline
	Проверить имущественное положение должника
		(розыск и арест счетов должника)
		& Проверка имущественного положения должника в принадлежащем ему
			или занимаемом им жилом помещении
		& Судебный пристав 
		& Инофрмация о счетах должника
		& Постановленипе о наложении ареста \\ \hline
	Отразить действие в документации
		& Составление документов ключающих результаты проверки имущественного
			положения должников и процесса возбуждение исполнительного
			производства
		& Судебный пристав 
		& Постановленипе о наложении ареста 
		& Постановление о реализации имущества \\ \hline
	Осуществить реализацию арестованного имущества
		& Принудительная реализация имущества должника осуществляется путем
			его продажи специализированными организациями,
			привлекаемыми в порядке
		& Судебный пристав 
		& Постановление о реализации имущества 
		& Документ на предостваление в банк \\ \hline
\end{longtable}
}

\leftline{Таблица 2. Таблично описание бизнес-процесса
	"<Проверить имущественное положение должника
	(розыск и арест счетов должника)">
	}
{\small
\begin{longtable}{
		|p{0.15\textwidth}
		|p{0.15\textwidth}
		|p{0.15\textwidth}
		|p{0.15\textwidth}
		|p{0.15\textwidth}
		| } 
	\hline
	\textbf{Название подпроцесса}
		& \textbf{Краткое описание}
		& \textbf{Исполнитель}
		& \textbf{Вход}
		& \textbf{Выход} \\ \hline
	\endhead
	Проверка сведений о должнике и имуществе
		& Осмотр жилого помещения приставами
			составление описи имущества
			и сравнение его с подоваемыми документами
		& Судебный пристав 
		& Инофрмация о счетах должника 
		& Сведенье об имуществе \\ \hline
	Оформление постановления на розыск
		& судебный пристав-исполнитель в ходе исполнительного
			производства объявляет исполнительный розыск должника,
			его имущества
		& Судебный пристав 
		& Сведенье об имуществе 
		& Постановление о розске \\ \hline
	Проведение розыскных мероприятий
		& Проведение опроса, наведения справок, исследования предметов
			и документов, отождествление личности и тд.
		& Судебный пристав 
		& Постановление о розске
		& Постановленипе о наложении ареста \\ \hline
\end{longtable}

\leftline{Таблица 3. Таблично описание бизнес-процесса
	"<Осуществить реализацию арестованного имущества">
	}
{\small
\begin{longtable}{
		|p{0.15\textwidth}
		|p{0.15\textwidth}
		|p{0.15\textwidth}
		|p{0.15\textwidth}
		|p{0.15\textwidth}
		| } 
	\hline
	\textbf{Название подпроцесса}
		& \textbf{Краткое описание}
		& \textbf{Исполнитель}
		& \textbf{Вход}
		& \textbf{Выход} \\ \hline
	\endhead
	Оценить имущество
		& Оценка имущества должника, на которое обращается взыскание,
			производится судебным приставом-исполнителем по рыночным ценам,
			если иное не установлено законодательством Российской Федерации.
		& Судебный пристав 
		& Постановление о реализации имущества
		& Оценочная стоимость имущества \\ \hline
	Выставить имущество на торги
		& Реализация на торгах имущества должника, в том числе имущественных
			прав, производится организацией или лицом,
			имеющими в соответствии с законодательством Российской Федерации
			право проводить торги по соответствующему виду имущества
		& Судебный пристав 
		& Оценочная стоимость имущества
		& Денежный эквивалент реализованного имущества \\ \hline
	Зачислить на счет подразделения
		& Зачисление на счет денежных средств с продажи имущества
		& Судебный пристав 
		& Денежный эквивалент реализованного имущества
		& Определенная сумма для погашения долга взыскателя
			и остаточная сумма реализации \\ \hline
	Закрыть долг взыскателя
		& Погашение задолжности
		& Судебный пристав 
		& Определенная сумма для погашения долга взыскателя
			и остаточная сумма реализации 
		& Документ на предостваление в банк,
			Сумма для погащения расходов \\ \hline
	Возместить расходы по совершению исполнительных действий
		& Возмещение расходов по совершению исполнительных действий
		& Судебный пристав 
		& Сумма для погащения расходов
		& Документ на предостваление в банк \\ \hline
\end{longtable}

\leftline{Таблица 4. Таблично описание подпроцесса
	"<Управлять информационным взаимодействием">
	}
{\small
\begin{longtable}{
		|p{0.12\textwidth}
		|p{0.12\textwidth}
		|p{0.12\textwidth}
		|p{0.12\textwidth}
		|p{0.12\textwidth}
		|p{0.12\textwidth}
		|p{0.12\textwidth}
		| } 
	\hline
	\textbf{Название подпроцесса}
		& \textbf{Краткое описание}
		& \textbf{Исполни-тель}
		& \textbf{Вход}
		& \textbf{От кого}
		& \textbf{Выход}
		& \textbf{Кому} \\ \hline
	\endhead
	Внести постановление о возбуждении исполнительного производства	
		& Исполнитель-ное производство возбуждается на основании
			исполнительного документа по заявлению взыскателя,
			если иное не установлено настоящим Федеральным законом
		& Судебный пристав 
		& Исполнительный лист 
		& 
		& Постановле-ние о возбуждении исполнительного производства
		& Проверить имущественное положение должника
			(розыск и арест счетов должника) \\ \hline
	Проверить имущественное положение должника
		(розыск и арест счетов должника)
		& Проверка имущественного положения должника в принадлежащем ему
			или занимаемом им жилом помещении
		& Судебный пристав 
		& Инофрмация о счетах должника
		& Внести постановление о возбуждении исполнительного производства
		& Постановленипе о наложении ареста
		& Отразить действие в документации \\ \hline
	Отразить действие в документации
		& Составление документов ключающих результаты проверки имущественного
			положения должников и процесса возбуждение исполнительного
			производства
		& Судебный пристав 
		& Постановленипе о наложении ареста 
		& Проверить имущественное положение должника
			(розыск и арест счетов должника)
		& Постановление о реализации имущества
		& Осуществить реализацию арестованного имущества\\ \hline
	Осуществить реализацию арестованного имущества
		& Принудительная реализация имущества должника осуществляется путем
			его продажи специализированными организациями,
			привлекаемыми в порядке
		& Судебный пристав 
		& Постановление о реализации имущества 
		& Отразить действие в документации
		& Документ на предостваление в банк
		& \\ \hline
\end{longtable}
}

\leftline{Таблица 5. Таблично описание подпроцесса
	"<Проверить имущественное положение должника
	(розыск и арест счетов должника)">
	}
{\small
\begin{longtable}{
		|p{0.12\textwidth}
		|p{0.12\textwidth}
		|p{0.12\textwidth}
		|p{0.12\textwidth}
		|p{0.12\textwidth}
		|p{0.12\textwidth}
		|p{0.12\textwidth}
		| } 
	\hline
	\textbf{Название подпроцесса}
		& \textbf{Краткое описание}
		& \textbf{Исполнитель}
		& \textbf{Вход}
		& \textbf{От кого}
		& \textbf{Выход}
		& \textbf{Кому} \\ \hline
	\endhead
	Проверка сведений о должнике и имуществе
		& Осмотр жилого помещения приставами
			составление описи имущества
			и сравнение его с подоваемыми документами
		& Судебный пристав 
		& Инофрмация о счетах должника 
		& Проведение розыскных мероприятий
		& Сведенье об имуществе
		& Оформление постановления на розыск \\ \hline
	Оформление постановления на розыск
		& судебный пристав-исполнитель в ходе исполнительного
			производства объявляет исполнительный розыск должника,
			его имущества
		& Судебный пристав 
		& Сведенье об имуществе 
		& Проверка сведений о должнике и имуществе
		& Постановление о розске
		& Проведение розыскных мероприятий  \\ \hline
	Проведение розыскных мероприятий
		& Проведение опроса, наведения справок, исследования предметов
			и документов, отождествление личности и тд.
		& Судебный пристав 
		& Постановление о розске
		& Отразить действие в документации
		& Постановленипе о наложении ареста
		&  \\ \hline
\end{longtable}

\leftline{Таблица 6. Таблично описание подпроцесса
	"<Осуществить реализацию арестованного имущества">
	}
{\small
\begin{longtable}{
		|p{0.12\textwidth}
		|p{0.12\textwidth}
		|p{0.12\textwidth}
		|p{0.12\textwidth}
		|p{0.12\textwidth}
		|p{0.12\textwidth}
		|p{0.12\textwidth}
		| } 
	\hline
	\textbf{Название подпроцесса}
		& \textbf{Краткое описание}
		& \textbf{Исполнитель}
		& \textbf{Вход}
		& \textbf{От кого}
		& \textbf{Выход}
		& \textbf{Кому} \\ \hline
	\endhead
	Оценить имущество
		& Оценка имущества должника, на которое обращается взыскание,
			производится судебным приставом-исполнителем по рыночным ценам,
			если иное не установлено законодательством Российской Федерации.
		& Судебный пристав 
		& Постановление о реализации имущества
		& Выставить имущество на торги
		& Оценочная стоимость имущества
		& Выставить имущество на торги \\ \hline
	Выставить имущество на торги
		& Реализация на торгах имущества должника, в том числе имущественных
			прав, производится организацией или лицом,
			имеющими в соответствии с законодательством Российской Федерации
			право проводить торги по соответствующему виду имущества
		& Судебный пристав 
		& Оценочная стоимость имущества
		& Оценить имущество
		& Денежный эквивалент реализованного имущества
		& Зачислить на счет подразделения \\ \hline
	Зачислить на счет подразделения
		& Зачисление на счет денежных средств с продажи имущества
		& Судебный пристав 
		& Денежный эквивалент реализованного имущества
		& Выставить имущество на торги
		& Определенная сумма для погашения долга взыскателя
			и остаточная сумма реализации
		& Закрыть долг взыскателя \\ \hline
	Закрыть долг взыскателя
		& Погашение задолжности
		& Судебный пристав 
		& Определенная сумма для погашения долга взыскателя
			и остаточная сумма реализации 
		& Зачислить на счет подразделения
		& Документ на предостваление в банк,
			Сумма для погащения расходов
		& Возместить расходы по совершению исполнительных действий \\ \hline
	Возместить расходы по совершению исполнительных действий
		& Возмещение расходов по совершению исполнительных действий
		& Судебный пристав 
		& Сумма для погащения расходов
		&
		& Документ на предостваление в банк
		&  \\ \hline
\end{longtable}

\section*{ВЫВОД}
\addcontentsline{toc}{section}{ВЫВОД}

Мы исправили семантические ошибки в контекстной диаграмме 
практической работы №4 и таблично описали декомпозированные блоки.
Получили построенные и сохраненные в
файле текстового формата:

\begin{itemize}
	\item измененная функциональная диаграмма с текстовым описанием
		изменений;
	\item построенный подпроцесс с табличным описанием (форма таблицы
		представлена далее)
\end{itemize}

\section*{СПИСОК ЛИТЕРАТУРЫ}
\addcontentsline{toc}{section}{СПИСОК ЛИТЕРАТУРЫ}
\begin{thebibliography}{}
    \bibitem{}  Материалы для практических/семинарских занятий: [url] 
		\url{https://online-edu.mirea.ru/mod/resource/view.php?id=496092}
\end{thebibliography}
